
\section*{Introduction}

We are Quantum\_Shadowcasters. In this document, we'll briefly discuss our team dynamics, and explain how we solved challenge 10B. % (Changing our team name from something AI generated was on our todo list, but somehow stuck it.)

Our team is comprised of five members, of which four participated in Cipher Challenge last year. We all attend Cambridge Maths School, a specialist maths sixth-form.

\subsection*{Team Members}

\begin{itemize}
    \item Erik Hurinek (Y13)
    \item Timur Perfilov (Y13)
    \item Rayyan Ansari (Y12)
    \item Louise Hardy (Y13)
    \item Samuel Maru (Y13)
\end{itemize}

\subsection*{Organisation}

To collaboratively work on challenges, we had Discord voice chats each Thursday evening at an agreed time. We used GitHub for source control and organisation of code and submissions. Our repository is organised into different sections for code, testing and submissions for each challenge. We also made use of GitHub Issues to manage and allocate tasks to be done.

We plan to make our GitHub repository publicly visible after the challenge. It can be found here: \url{https://github.com/ErikPeter2000/cipherchallenge2024}. The link may not work until the repository is made public.

\subsection*{Software Used}

Before the challenge began, we made a list of ciphers frequently used in previous challenges and began developing programs to solve them. Many of these programs used brute-force and evolutionary algorithms, such as hill-climbing methods. To ensure accuracy within these algorithms, we wrote unit tests to test our code with data from past challenges. These main algorithms were written in Scala, although each team member mostly used the language they were most familiar with. These included Java, JavaScript and Python.