
\section*{Introduction}

We are AES Alchemists. In this document, we will briefly discuss our team dynamics and explain how we solved challenge 10B of the National Cipher Challenge.

Our team consists of five members, of which four participated in Cipher Challenge last year. We all attend Cambridge Maths School, a specialist maths sixth form.

\subsection*{Team Members}

\begin{itemize}
    \item Erik Hurinek (Y13)
    \item Timur Perfilov (Y13)
    \item Rayyan Ansari (Y12)
    \item Louise Hardy (Y13)
    \item Samuel Maru (Y13)
\end{itemize}

\subsection*{Organisation}

To collaboratively work on challenges, we called each Thursday night at an agreed time via Discord. We used GitHub for source control and organisation of code. Our repository is organised into different sections for code, testing, and submissions for each challenge. We also used GitHub Issues to manage and allocate tasks to be done.

We plan to make our GitHub repository public after the challenge, to allow others to see our code and solutions. Once published, it will be found here: \url{https://github.com/ErikPeter2000/cipherchallenge2024}.

\subsection*{Software Used}

Before the challenge began, we made a list of ciphers frequently used in previous challenges and began developing programs to solve them. Many of these programs used brute-force and evolutionary algorithms, such as hill-climbing methods. To ensure accuracy within these algorithms, we wrote unit tests to test our code with data from previous challenges. Our main programs were written in Scala, although team members sometimes used the language they were most familiar with. These included Java and Python.